\documentclass[]{article}
\usepackage{lmodern}
\usepackage{amssymb,amsmath}
\usepackage{ifxetex,ifluatex}
\usepackage{fixltx2e} % provides \textsubscript
\ifnum 0\ifxetex 1\fi\ifluatex 1\fi=0 % if pdftex
  \usepackage[T1]{fontenc}
  \usepackage[utf8]{inputenc}
\else % if luatex or xelatex
  \ifxetex
    \usepackage{mathspec}
  \else
    \usepackage{fontspec}
  \fi
  \defaultfontfeatures{Ligatures=TeX,Scale=MatchLowercase}
\fi
% use upquote if available, for straight quotes in verbatim environments
\IfFileExists{upquote.sty}{\usepackage{upquote}}{}
% use microtype if available
\IfFileExists{microtype.sty}{%
\usepackage[]{microtype}
\UseMicrotypeSet[protrusion]{basicmath} % disable protrusion for tt fonts
}{}
\PassOptionsToPackage{hyphens}{url} % url is loaded by hyperref
\usepackage[unicode=true]{hyperref}
\hypersetup{
            pdfborder={0 0 0},
            breaklinks=true}
\urlstyle{same}  % don't use monospace font for urls
\usepackage[margin=2cm]{geometry}
\usepackage{longtable,booktabs}
% Fix footnotes in tables (requires footnote package)
\IfFileExists{footnote.sty}{\usepackage{footnote}\makesavenoteenv{long table}}{}
\IfFileExists{parskip.sty}{%
\usepackage{parskip}
}{% else
\setlength{\parindent}{0pt}
\setlength{\parskip}{6pt plus 2pt minus 1pt}
}
\setlength{\emergencystretch}{3em}  % prevent overfull lines
\providecommand{\tightlist}{%
  \setlength{\itemsep}{0pt}\setlength{\parskip}{0pt}}
\setcounter{secnumdepth}{0}
% Redefines (sub)paragraphs to behave more like sections
\ifx\paragraph\undefined\else
\let\oldparagraph\paragraph
\renewcommand{\paragraph}[1]{\oldparagraph{#1}\mbox{}}
\fi
\ifx\subparagraph\undefined\else
\let\oldsubparagraph\subparagraph
\renewcommand{\subparagraph}[1]{\oldsubparagraph{#1}\mbox{}}
\fi

% set default figure placement to htbp
\makeatletter
\def\fps@figure{htbp}
\makeatother

\usepackage{mhchem}

\date{}

\begin{document}

\subsection{Equilbrium}\label{equilbrium}

\hfill{Sasha Hydrie}

\paragraph{Terms:}\label{terms}

\begin{itemize}
\tightlist
\item
  equilibrium eonstant: \(K\)
\end{itemize}

\begin{quote}
\begin{quote}
Calculated in the same way as \(Q\), but only at equilibrium conditions.
\end{quote}
\end{quote}

\begin{itemize}
\tightlist
\item
  reaction quotient: \(Q\)
\end{itemize}

\begin{quote}
\begin{quote}
For a reaction \ce{aA + bB -> cC + dD}, \(Q=\dfrac{C^cD^d}{A^aB^b}\).
Remember K to Q comparison, if in alphabetical order the sign points, K
\textgreater{} Q, reaction favors right.
\end{quote}
\end{quote}

\begin{itemize}
\tightlist
\item
  gas pressure constant: \(K_p\)
\end{itemize}

\begin{quote}
\begin{quote}
Related to \(K_c\) by the formula \(K_p=K_c(RT)^{\Delta n}\), given
\(\Delta n\) is the difference in moles of gas, products - reactants,
\(R\) is the ideal gas constant, and \(T\) is temperature in Kelvin.
\end{quote}
\end{quote}

\begin{itemize}
\tightlist
\item
  acid dissociation constant: \(K_a\), larger means stronger acid and
  more dissociation, only for weak acids since strong dissociate
  entirely.
\end{itemize}

\begin{quote}
\begin{quote}
Derived from the reaction \ce{HA -> H^{+} + A^{-}},
\(K_a=\dfrac{\ce{\lbrack H^{+}\rbrack \lbrack A^{-}\rbrack }}{\ce{\lbrack HA\rbrack }}\)
equivalent is \ce{HA + H2O -> H3O^{+} + A-}. Also,
\(\text{p}K_a = -\log{K_a}\)
\end{quote}
\end{quote}

\begin{itemize}
\tightlist
\item
  base dissociation constant: \(K_b\)
\end{itemize}

\begin{quote}
\begin{quote}
Derived from the reaction \ce{HB + H2O -> OH^{-} + BH^{+}},
\(K_a=\dfrac{\lbrack OH^{-}\rbrack \lbrack BH^{+}\rbrack }{\lbrack B\rbrack }\)
Also, \(\text{p}K_b = -\log{K_b}\)
\end{quote}
\end{quote}

\begin{itemize}
\tightlist
\item
  autoionzation: the process of \ce{2H2O -> H3O^{+} + OH^{-}}
\item
  Ion product consant: \(K_w=10^{-14}=K_aK_b\)
\end{itemize}

\begin{quote}
\begin{quote}
Equal to \ce{\lbrack H3O^{+}\rbrack \lbrack OH^{-}\rbrack }, hence if
either ion is known pH can be derived.
\end{quote}
\end{quote}

\begin{itemize}
\item
  weak acid: only partially dissociates, proton donor
\item
  strong acid: completely dissociates, proton donor
\item
  weak base: partially dissociates, proton acceptor
\item
  strong base: completely dissociates, proton acceptor
\item
  conjugate acid base-pairs: differ only by a singel proton, ex:
  \ce{NH3} and \ce{NH4+}
\item
  polyprotic acids: acids that can donate multiple protons, \(K_a\) for
  the first dissociation is usually the only useful one.
\item
  amphoteric: can function as an acid or base, depending on species. ex-
  water autoionzation.
\item
  solubility product constant: \(K_{sp}\) product of ion concentrations.
\end{itemize}

\begin{quote}
\begin{quote}
\(K_{sp}=\text{\lbrack Cation\rbrack \lbrack Anion\rbrack }\), the solid
is excluded since its concentration is (by definition) 1.
\end{quote}
\end{quote}

\paragraph{Le Chatelier's Principle:}\label{le-chateliers-principle}

\textbf{Think about value of Q for each}

\begin{enumerate}
\def\labelenumi{\arabic{enumi}.}
\tightlist
\item
  Concentration increase → favors opposite side
\item
  Pressure increase → favors side with fewer moles of \emph{gas}
\item
  Temperature increase → favores endothermic direction
\item
  Catalyst → \emph{Nope!} (catalyst speeds up mechanism both ways)
\item
  Completely remove solid → favors side with removed solid
\item
  Partially remove solid → \emph{Nope!}
\end{enumerate}

\newpage 

\paragraph{Acids and Bases:}\label{acids-and-bases}

Reactions involving weak acids or bases and water are \emph{reversible},
while reactions involving strong acids or bases are \emph{not}. As an
example,

\begin{quote}
\ce{HCl + H2O -> Cl^{-} + H3O^{+}}
\end{quote}

note the usage of the single arrow. Comparitively,

\begin{quote}
\ce{CH3COOH + H2O <-> H3O^{+} + CH3COO^{-}}.
\end{quote}

Since weak acids and bases do not dissociate completely there are
factors that effect how much they dissociate. An example would be the
\textbf{Common Ion Effect:}

\begin{quote}
weak electrolytes dissociate less when there is a shared ion present, Le
Chatelier's can be used as justification, looking at the net ionic
equation.
\end{quote}

A strong base and a strong acid reacting in equal quantity yields a
neutral solution:

\begin{quote}
\ce{HCl + NaOH -> NaCl + H2O}
\end{quote}

A strong acid reacting with a weak base in equal quantity yields a weak
acid, making the solution acidic overall:

\begin{quote}
\ce{HCl + CH3COONa <-> NaCl + CH3COOH}, the reaction is reversible
because the net ionic equation is \ce{H+ + CH3COO- -> CH3COOH}.
\end{quote}

The converse is true when a strong base and a weak acid react. Weak
acids do not react appreciably with weak bases, even in significant
concentrations. (See \textbf{Buffers}.)

\paragraph{Salts:}\label{salts}

The reaction of an acid and base produces a salt, the anion from the
acid with the cation from the base. Inherently, cations are proton
donors and anions are proton acceptors, making them acids and bases
respectively.

They can be considered conjugate bases or acids, and since they
correspond to strong acids or bases respectfully they do not dissociate
meaningfully, and are neutral. The neutral cations are

\begin{quote}
\ce{Li+}, \ce{Na+}, \ce{K+}, \ce{Rb+}, \ce{Cs+}, \ce{Ca^{2+}}, and
\ce{Sr^{2+}}
\end{quote}

and the neutral anions are

\begin{quote}
\ce{Cl-}, \ce{Br-}, \ce{I-}, \ce{NO3-}, \ce{ClO3-}, and \ce{ClO4-}.
\end{quote}

To determine if a salt is acidic or basic look at the components:

\begin{longtable}[]{@{}llr@{}}
\toprule
Cation & Anion & Salt\tabularnewline
\midrule
\endhead
Strong Base & Strong Acid & Neutral\tabularnewline
Strong Base & Weak Acid & Basic\tabularnewline
Weak Base & Strong Acid & Acidic\tabularnewline
Weak Base & Weak Acid & \(K_a\) vs \(K_b\)\tabularnewline
\bottomrule
\end{longtable}

\paragraph{Buffers:}\label{buffers}

Buffers are substances that resist change in pH. There are two types of
buffers:

\begin{enumerate}
\def\labelenumi{\arabic{enumi}.}
\tightlist
\item
  Weak acid + conjugate base, ex-\ce{HF} and \ce{F-}
\end{enumerate}

\begin{quote}
\begin{quote}
The weak acid neutralizes any base added and the conjugate base
neutralizes acid.
\end{quote}
\end{quote}

\begin{enumerate}
\def\labelenumi{\arabic{enumi}.}
\setcounter{enumi}{1}
\tightlist
\item
  Weak base + conjugate acid, ex-\ce{NH3} and \ce{NH4+}
\end{enumerate}

\begin{quote}
\begin{quote}
The weak base neutralizes any acid added and the conjugate acid
neutralizes base.
\end{quote}
\end{quote}

A common misconception is that a buffer can not be prepared with a
strong base or acid, but if the weak counterpart is in excess then it
will produce its conjugate while ultimately using all of the strong
substance.

To calculate the pH of a buffer, use ICE as if calculating the pH of a
weak acid to determine \(K_a\). Left as an exercise to the reader.

\newpage 

It's unlikely that any reader did the aforementioned exercise, but the
end result is the equation

\begin{quote}
\(K_a = x\dfrac{\text{\lbrack base\rbrack }}{\text{\lbrack acid\rbrack }}\),
which written in logarithmic form yields\ldots{}
\end{quote}

The \textbf{Henderson-Hasselbalch equation}:
\[\text{pH} = \text{p}K_a + \log{\dfrac{\ce{\lbrack A-\rbrack }}{\ce{\lbrack HA\rbrack }}},\]

similarly,

\[\ce{\lbrack OH-\rbrack } = K_b\times \dfrac{\ce{\lbrack B\rbrack }}{\ce{\lbrack HB+\rbrack }}\text{ and } \text{pOH}=\text{p}K_b + \log{\dfrac{\ce{\lbrack HB+\rbrack }}{\ce{\lbrack B\rbrack }}}.\]

Buffer capacity is based on the amount of strong acid or base that a
buffer can absorb without a large pH change. Generally a buffer is only
functional within a range of a buffer is within 0.5 pH of the p\(K_a\)
value.

\paragraph{Titration:}\label{titration}

A titration is a procedure used to determine the concentration of an
unknown solution. An \textbf{indicator} is used to signal when the
equivalence point, when the solution has reached a certain pH threshold,
is reached. The component added is the \textbf{titrant}.

A good rule of thumb is that strong acids have a pH less than two and
strong bases have a pH greater than twelve. If a strong base is titrated
with a weak acid or the converse then the equivalence point will be at 7
pH. Otherwise, it's skewed towards the pH of the strong.

\paragraph{Solubility}\label{solubility}

There is not too much to this topic beyond the definition. However, when
there are coefficients concentrations are exponential, as in all forms
of equilibrium. As an example, \ce{MgF2} would have a constant of

\begin{quote}
\begin{quote}
\(K_{sp} =\)\lbrack \ce{Mg^{2+}}\rbrack \lbrack \ce{F-}\rbrack \(^2\)
\end{quote}
\end{quote}

\end{document}
